\documentclass[letterpaper,twocolumn,10pt]{article}

\usepackage[letterpaper, margin=1in, tmargin=1.5in]{geometry}
\usepackage{fancyhdr}
\usepackage{microtype}

\usepackage{xcolor}
\usepackage{listings}
\usepackage{booktabs}

\definecolor{lightgrey}{rgb}{0.95,0.95,0.95}

% VS2017 C++ color scheme
\definecolor{clr-background}{RGB}{255,255,255}
\definecolor{clr-text}{RGB}{0,0,0}
\definecolor{clr-string}{RGB}{163,21,21}
\definecolor{clr-namespace}{RGB}{0,0,0}
\definecolor{clr-preprocessor}{RGB}{128,128,128}
\definecolor{clr-keyword}{RGB}{0,0,255}
\definecolor{clr-type}{RGB}{43,145,175}
\definecolor{clr-variable}{RGB}{0,0,0}
\definecolor{clr-constant}{RGB}{111,0,138} % macro color
\definecolor{clr-comment}{RGB}{0,128,0}

\lstdefinestyle{CStyle}{
    language=C,
    % VS colors
    backgroundcolor=\color{clr-background},
    basicstyle=\small\ttfamily\color{clr-text},
    stringstyle=\color{clr-string},
    identifierstyle=\color{clr-variable},
    commentstyle=\color{clr-comment},
    directivestyle=\color{clr-preprocessor},
    keywordstyle=\color{clr-type},
    keywordstyle={[2]\color{clr-constant}},
    %frame=tb,
    captionpos=b,
    columns=fullflexible,
    %backgroundcolor=\color{lightgrey},
    showstringspaces=false,
    keepspaces=true,
    tabsize=8,
    numbers=left,
    numbersep=5pt,
    linewidth=0.7\textwidth
}

\lstset{style=CStyle}

\pagestyle{fancy}
\fancyhead[L]{\bf CS350: Operating Systems}
\fancyfoot[C]{\thepage}



\fancyhead[R]{\bf Paging}

\begin{document}

Consider a paging-based virtual memory system with 32-bit virtual and physical 
adresses, and a page size of $2^{12}$ bytes (4~KiB).  Suppose that a process P 
is running. P uses only 128~KiB of virtual memory.  The first 5 entries of P's 
page table are shown below.

\begin{table}
\centering
\begin{tabular}{l|l|c}
\toprule
Page \# & Frame \# & Valid \\
\midrule
0x0 & 0x00234 & 1\\
    0x1 & 0x00235 & 1\\
    0x2 & 0x0023f & 1\\
    0x3 & 0x00ace & 1\\
    0x4 & 0x00004 & 1\\
\bottomrule
\end{tabular}
\end{table}

\vspace{4em}

\noindent
\textbf{Question 1.} What is the total number of entries in P's page table?

\vspace{16em}

\noindent
\textbf{Question 2.} How many entries are valid?

\vspace{16em}
\break

\textbf{Question 3.} What physical addresses correspond to each of these 
virtual addresses?

\begin{itemize}
\item 0x00001a60
\item 0x00000fb5
\item 0x00004664
\end{itemize}

\vspace{16em}

\textbf{Question 4.} If the page size were 16~KiB instead of 4~KiB, how many 
entries would there be in P's page table?  How many bits of each virtual 
address would be used for the offset, and how many for the page number?

\end{document}

