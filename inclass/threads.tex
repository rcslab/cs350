\documentclass[letterpaper,twocolumn,10pt]{article}

\input{macros}

\fancyhead[R]{\bf Threads}

\begin{document}

\begin{lstlisting}[style=CStyle]
#define NTHR 8
#define NARR (1024*1024)
int array[NARR];

typedef struct Args {
    int start; int end; int max;
} Args;

Args *
search(Args *args)
{
    args->max = array[0];
    for (int i = args->start;
	 i < args->end;
	 i++) {
	if (args->max < array[i])
	    args->max = array[i];
    }

    return args;
}

int
main(int argc, const char *argv[])
{
    pthread_t thr[NTHR];
    Args args[NTHR];
    
    void *rval;

    // Fill in random values
    arc4random_buf(&array[0], sizeof(array));

    for (int i = 0; i < NTHR; i++) {
	args[i].start = NARR/NTHR*i;
	args[i].end = NARR/NTHR*(i+1);
	pthread_create(&thr[i], NULL, &search, &args[i]);
    }

    for (int i = 0; i < NTHR; i++) {
	pthread_join(thr[i], &rval);
    }

    int max = args[0].max;
    for (int i = 0; i < NTHR; i++) {
	if (max < args[i].max)
	    max = args[i].max;
    }

    printf("Max: %d, Max (hex): %08x\n", max, max);

    return 0;
}
\end{lstlisting}

\break

\noindent
\textbf{Question 1.} What does the function do?

\vspace{12em}

\noindent
\textbf{Question 2.} If you have eight cores what is the optimal value for 
NTHR? Why?

\vspace{12em}

\noindent
\textbf{Question 3.} If you have 1024 cores what is the optimal value for NTHR?  
Why?

\vspace{12em}
\noindent
\textbf{Question 4.} What is the expected output (give a number)?

\vspace{12em}

\end{document}

