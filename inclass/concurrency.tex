\documentclass[letterpaper,twocolumn,10pt]{article}

\usepackage[letterpaper, margin=1in, tmargin=1.5in]{geometry}
\usepackage{fancyhdr}
\usepackage{microtype}

\usepackage{xcolor}
\usepackage{listings}
\usepackage{booktabs}

\definecolor{lightgrey}{rgb}{0.95,0.95,0.95}

% VS2017 C++ color scheme
\definecolor{clr-background}{RGB}{255,255,255}
\definecolor{clr-text}{RGB}{0,0,0}
\definecolor{clr-string}{RGB}{163,21,21}
\definecolor{clr-namespace}{RGB}{0,0,0}
\definecolor{clr-preprocessor}{RGB}{128,128,128}
\definecolor{clr-keyword}{RGB}{0,0,255}
\definecolor{clr-type}{RGB}{43,145,175}
\definecolor{clr-variable}{RGB}{0,0,0}
\definecolor{clr-constant}{RGB}{111,0,138} % macro color
\definecolor{clr-comment}{RGB}{0,128,0}

\lstdefinestyle{CStyle}{
    language=C,
    % VS colors
    backgroundcolor=\color{clr-background},
    basicstyle=\small\ttfamily\color{clr-text},
    stringstyle=\color{clr-string},
    identifierstyle=\color{clr-variable},
    commentstyle=\color{clr-comment},
    directivestyle=\color{clr-preprocessor},
    keywordstyle=\color{clr-type},
    keywordstyle={[2]\color{clr-constant}},
    %frame=tb,
    captionpos=b,
    columns=fullflexible,
    %backgroundcolor=\color{lightgrey},
    showstringspaces=false,
    keepspaces=true,
    tabsize=8,
    numbers=left,
    numbersep=5pt,
    linewidth=0.7\textwidth
}

\lstset{style=CStyle}

\pagestyle{fancy}
\fancyhead[L]{\bf CS350: Operating Systems}
\fancyfoot[C]{\thepage}



\fancyhead[R]{\bf Concurrency}

\begin{document}

A common design pattern in systems programming is to use a thread pool to farm 
out the processing of incoming tasks and scale up to use as many processors as 
available.  Each task is enqueued into a queue and processed by a worker 
thread.

\begin{lstlisting}[style=CStyle]
typedef struct Task {
    void (*func)(void *);
    void *arg;
} Task;

Task q[QUEUE_SIZE];
int in = 0, out = 0, count = 0;
bool exitRequested = false;
pthread_mutex_t m = PTHREAD_MUTEX_INITIALIZER;
pthread_cond_t c = PTHREAD_COND_INITIALIZER;

void
enqueue(void (*func)(void *), void *arg) {
    Task *t = malloc(sizeof(*t));

    t->func = func;
    t->arg = arg;

    q[in] = t;
    in = (in + 1) % BUFFER_SIZE;
    count++;

    pthread_cond_signal(&c);

}

void
workerthread(void *ignored) {
    for (;;) {
	pthread_mutex_lock(&m);

	if (count == 0)
	    pthread_cond_wait(&c, &m);
	if (exitRequested)
	    pthread_exit(NULL);

	Task *t = q[out];
	out = (out + 1) % BUFFER_SIZE;
	count--;
	pthread_mutex_unlock(&m);

	t->func(t->arg);

	free(t);
    }
}

\end{lstlisting}

\break

\noindent
\textbf{Question 1.} Does the program have any data races?  If so, explain the 
data race and fix the code.

\vspace{16em}

\noindent
\textbf{Question 2.} Are the condition variables used correctly?  If not, 
explain the bug(s) and fix the code.

\vspace{16em}

\textbf{Question 3.} Write a function to terminate the worker pool.  Hint: You 
may need to fix the code to left.

\vspace{16em}

\end{document}

