\documentclass[letterpaper,twocolumn,10pt]{article}

\usepackage[letterpaper, margin=1in, tmargin=1.5in]{geometry}
\usepackage{fancyhdr}
\usepackage{microtype}

\usepackage{xcolor}
\usepackage{listings}
\usepackage{booktabs}

\definecolor{lightgrey}{rgb}{0.95,0.95,0.95}

% VS2017 C++ color scheme
\definecolor{clr-background}{RGB}{255,255,255}
\definecolor{clr-text}{RGB}{0,0,0}
\definecolor{clr-string}{RGB}{163,21,21}
\definecolor{clr-namespace}{RGB}{0,0,0}
\definecolor{clr-preprocessor}{RGB}{128,128,128}
\definecolor{clr-keyword}{RGB}{0,0,255}
\definecolor{clr-type}{RGB}{43,145,175}
\definecolor{clr-variable}{RGB}{0,0,0}
\definecolor{clr-constant}{RGB}{111,0,138} % macro color
\definecolor{clr-comment}{RGB}{0,128,0}

\lstdefinestyle{CStyle}{
    language=C,
    % VS colors
    backgroundcolor=\color{clr-background},
    basicstyle=\small\ttfamily\color{clr-text},
    stringstyle=\color{clr-string},
    identifierstyle=\color{clr-variable},
    commentstyle=\color{clr-comment},
    directivestyle=\color{clr-preprocessor},
    keywordstyle=\color{clr-type},
    keywordstyle={[2]\color{clr-constant}},
    %frame=tb,
    captionpos=b,
    columns=fullflexible,
    %backgroundcolor=\color{lightgrey},
    showstringspaces=false,
    keepspaces=true,
    tabsize=8,
    numbers=left,
    numbersep=5pt,
    linewidth=0.7\textwidth
}

\lstset{style=CStyle}

\pagestyle{fancy}
\fancyhead[L]{\bf CS350: Operating Systems}
\fancyfoot[C]{\thepage}



\fancyhead[R]{\bf User Level Threads}

\begin{document}

\begin{lstlisting}[style=CStyle]
#define NFIBERS 4

ucontext_t fibers[NFIBERS];
int done[NFIBERS];
int current;

void fiber_yield(void) {
    int prev = current;

    for (int i = 0; i < NFIBERS; ++i) {
        current = (current + 1) % NFIBERS;

        if (!done[current]) {
            if (current != prev)
                swapcontext(&fibers[prev],
                    &fibers[current]);
            break;
        }
    }
}

void fiber_exit(void) {
    done[current] = 1;
    fiber_yield();
}

void fiber_fn(int start, int end) {
    int sum = 0;

    for (int i = start; i < end; ++i) {
        sum += i;
        if (i == (start + end) / 2) {
            printf("sum: %d (yield)\n", sum);
            fiber_yield();
        }
    }

    printf("sum: %d\n", sum);
    fiber_exit();
}

int main(void) {
    for (int i = 0; i < NFIBERS; ++i) {
        // Initialize fiber[i] and allocate stack
        getcontext(&fibers[i]);
        fibers[i].uc_stack.ss_sp = malloc(SIGSTKSZ);
        fibers[i].uc_stack.ss_size = SIGSTKSZ;
        fibers[i].uc_link = NULL;
        makecontext(&fibers[i],
            (void (*)())fiber_fn,
            2, 100 * i, 100 * (i + 1));
    }

    setcontext(&fibers[0]);
}
\end{lstlisting}

\break

\noindent

Fibers are a name for lightweight user level threads.
\texttt{ucontext\_t} holds a fiber context.

\vspace{1em}

\texttt{makecontext(fiber, fn, argc, ...)} creates a fiber that runs the function \texttt{fn}, passing it \texttt{argc} arguments.

\vspace{1em}

\texttt{swapcontext(prev, next)} suspends the \texttt{prev} fiber and activates \texttt{next}.

\vspace{2em}

\textbf{Question 1.} Is this code with 4 fibers likely to run faster or slower than single-threaded code?

\vspace{12em}

\noindent
\textbf{Question 2.} How many context switches between fibers will there be?

\vspace{12em}
\noindent
\textbf{Question 3.} What is the expected output?

\vspace{12em}

\end{document}

